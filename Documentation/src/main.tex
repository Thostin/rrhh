\documentclass[a4paper]{scrartcl}

\usepackage[spanish, shorthands=off, es-nosectiondot, es-nolists]{babel}
\usepackage[sexy]{evanspanish}
\usepackage{comment}

\hypersetup{unicode,
  pdftitle={Documentación de Programa de Recursos Humanos},
  pdfauthor={Ryoma y Ever},
  pdfsubject={Documentación},
  pdfkeywords={},
  pdfproducer={},
  pdfcreator={pdflatex+evan.sty}
}

\ihead{\footnotesize\textbf{Documentación}}
\ohead{\footnotesize Recursos Humanos}

\usepackage{amsmath, amsfonts, amssymb, amsthm, comment, xcolor, svg, float, cancel}
\usepackage[top=1in, bottom=1in]{geometry}


\usepackage{mathtools}
\newcommand{\paren}[1]{\left( #1 \right)}

%\setcounter{secnumdepth}{2}
%\setcounter{tocdepth}{1}


\begin{document}
\title{Documentación\\RRHH}
\author{Nico Paz}
\date{\today}
\maketitle
\tableofcontents
\clearpage

\section{Introducción}
El software de gestión de horarios de funcionarios de
\textit{Recursos Humanos} permite a los operarios
de este departamento controlar los ingresos y salidas
de todos los funcionarios del colegio, así como elaborar
un detallado reporte de asistencias y recuento de horas
de presencia.

\clearpage
\section{Botones}
\subsection{Funcionarios}
Aquí se puede visualizar una lista completa de todos los funcionarios que
han sido añadidos a la base de datos. Se puede añadir, modificar o eliminar
funcionarios.
\subsubsection{Ver horario}
Se puede hacer click encima de un funcionario y
después hacer click en el botón \texttt{Ver horario}.
En esta nueva ventana se puede ver una lista de
horarios que posee ese funcionario a lo largo de
una semana; aquí también se puede añadir, modificar
y eliminar horario.

Al añadir un horario, se puede especificar con una casilla
si el horario es compensado o no compensado.

\subsection{Salas}
A través de este botón se pueden añadir, modificar y eliminar salas del
colegio.

\subsection{Chequear}
Aquí se puede seleccionar un funcionario y un rango de fechas
(posiblemente un solo día) y con el botón \texttt{Revisar}
se genera un reporte del funcionario con sus asistencias y
horarios correspondientes a ese rango de fechas.

\begin{comment}
  TODO: Citar las reglas con que se consideran las faltas
  y como se cuentan las horas

\end{comment}

\subsection{Ayuda}
Abrir esta guía del usuario.
\end{document}
