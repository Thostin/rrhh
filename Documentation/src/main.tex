\documentclass[a4paper]{scrartcl}

\usepackage[spanish, shorthands=off, es-nosectiondot, es-nolists]{babel}
\usepackage[sexy]{evanspanish}
\usepackage{comment}

\hypersetup{unicode,
  pdftitle={Documentación de Programa de Recursos Humanos},
  pdfauthor={Ryoma y Ever},
  pdfsubject={Documentación},
  pdfkeywords={},
  pdfproducer={},
  pdfcreator={pdflatex+evan.sty}
}

\ihead{\footnotesize\textbf{Documentación}}
\ohead{\footnotesize Recursos Humanos}

\usepackage{amsmath, amsfonts, amssymb, amsthm, comment, xcolor, svg, float, cancel}
\usepackage[top=1in, bottom=1in]{geometry}


\usepackage{mathtools}
\newcommand{\paren}[1]{\left( #1 \right)}

%\setcounter{secnumdepth}{2}
%\setcounter{tocdepth}{1}


\begin{document}
\title{Documentación\\RRHH}
\author{Nico Paz}
\date{\today}
\maketitle
\tableofcontents
\clearpage

\section{Introducción}

El \textbf{software de de gestión de Recursos Humanos} (RRHH) está diseñado para optimizar la gestión diaria de las farmacias, facilitando
el control de inventarios, la atención a los clientes y la administración de las ventas y compras de
medicamentos. Esta herramienta digital centraliza y automatiza procesos como la facturación, el
seguimiento de stock, la consulta de historial de productos, y la generación de reportes financieros y
operativos.
Nuestro software garantiza una mayor precisión en la administración de medicamentos, reduciendo
errores comunes y permitiendo a los farmacéuticos dedicar más tiempo a la atención al cliente. Al
ser una plataforma intuitiva y flexible, se adapta a las necesidades de farmacias de cualquier
tamaño, mejorando la eficiencia operativa y contribuyendo a un servicio de salud más ágil y seguro.


El software de farmacia es una solución integral creada para mejorar la eficiencia en la gestión de
farmacias, automatizando procesos clave y brindando una plataforma intuitiva y accesible para los
farmacéuticos. Este sistema permite un control total sobre las operaciones de la farmacia, como la
administración de inventarios, la gestión de ventas, la facturación y el seguimiento de clientes y
proveedores.
Entre sus características principales se incluyen:
\begin{itemize}
	\ii Gestión de inventarios: Mantiene un registro actualizado de los medicamentos y productos
	disponibles, alertando sobre niveles bajos de stock y fechas de vencimiento.
	\ii Control de ventas: Facilita la venta de medicamentos, ofreciendo una interfaz de punto de
	venta (POS) ágil y fácil de usar, que permite emitir facturas y recibos en tiempo real.
	\ii Historial de clientes: Guarda y gestiona información de los clientes, como historial de
	compras y recetas, permitiendo un seguimiento detallado de las necesidades recurrentes.
	\ii Reportes y análisis: Genera reportes detallados sobre ventas, compras, y stock, ayudando a
	los administradores a tomar decisiones informadas y estratégicas para el negocio.
	\ii Integración con proveedores: Simplifica la relación con los proveedores, permitiendo
	realizar pedidos de manera automática o manual y mantener registros de compras.
	Este software está diseñado para mejorar la eficiencia y precisión en las farmacias, reduciendo el
	tiempo dedicado a tareas administrativas y minimizando errores humanos. Además, ofrece una
	experiencia de usuario amigable y personalizable, adaptable a las necesidades específicas de cada
	farmacia.
\end{itemize}
\clearpage

\end{document}
